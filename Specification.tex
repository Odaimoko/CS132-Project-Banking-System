\documentclass[]{article}
\usepackage{lmodern}
\usepackage{amssymb,amsmath}
\usepackage{ifxetex,ifluatex}
\usepackage{fixltx2e} % provides \textsubscript
\ifnum 0\ifxetex 1\fi\ifluatex 1\fi=0 % if pdftex
  \usepackage[T1]{fontenc}
  \usepackage[utf8]{inputenc}
\else % if luatex or xelatex
  \ifxetex
    \usepackage{mathspec}
  \else
    \usepackage{fontspec}
  \fi
  \defaultfontfeatures{Ligatures=TeX,Scale=MatchLowercase}
\fi
% use upquote if available, for straight quotes in verbatim environments
\IfFileExists{upquote.sty}{\usepackage{upquote}}{}
% use microtype if available
\IfFileExists{microtype.sty}{%
\usepackage[]{microtype}
\UseMicrotypeSet[protrusion]{basicmath} % disable protrusion for tt fonts
}{}
\PassOptionsToPackage{hyphens}{url} % url is loaded by hyperref
\usepackage[unicode=true]{hyperref}
\hypersetup{
            pdfborder={0 0 0},
            breaklinks=true}
\urlstyle{same}  % don't use monospace font for urls
\usepackage{longtable,booktabs}
% Fix footnotes in tables (requires footnote package)
\IfFileExists{footnote.sty}{\usepackage{footnote}\makesavenoteenv{long table}}{}
\usepackage{graphicx,grffile}
\makeatletter
\def\maxwidth{\ifdim\Gin@nat@width>\linewidth\linewidth\else\Gin@nat@width\fi}
\def\maxheight{\ifdim\Gin@nat@height>\textheight\textheight\else\Gin@nat@height\fi}
\makeatother
% Scale images if necessary, so that they will not overflow the page
% margins by default, and it is still possible to overwrite the defaults
% using explicit options in \includegraphics[width, height, ...]{}
\setkeys{Gin}{width=\maxwidth,height=\maxheight,keepaspectratio}
\IfFileExists{parskip.sty}{%
\usepackage{parskip}
}{% else
\setlength{\parindent}{0pt}
\setlength{\parskip}{6pt plus 2pt minus 1pt}
}
\setlength{\emergencystretch}{3em}  % prevent overfull lines
\providecommand{\tightlist}{%
  \setlength{\itemsep}{0pt}\setlength{\parskip}{0pt}}
\setcounter{secnumdepth}{0}
% Redefines (sub)paragraphs to behave more like sections
\ifx\paragraph\undefined\else
\let\oldparagraph\paragraph
\renewcommand{\paragraph}[1]{\oldparagraph{#1}\mbox{}}
\fi
\ifx\subparagraph\undefined\else
\let\oldsubparagraph\subparagraph
\renewcommand{\subparagraph}[1]{\oldsubparagraph{#1}\mbox{}}
\fi

% set default figure placement to htbp
\makeatletter
\def\fps@figure{htbp}
\makeatother


\date{}

\begin{document}

\section{Specification}\label{header-n90}

This document will explain which part of Requirement document is
implemented in the code.

\textbf{Open account}. The following image is the use case of the
procedure open account, depicted in the requirement document. Here we
only have the clerk UI so the process from the customer to the clerk is
omitted.

\begin{figure}
\centering
\includegraphics{/Users/oda/Documents/iMyPRJ/CS132-Project-Banking-System/assets/image-20190513222322003.png}
\caption{}
\end{figure}

\begin{figure}
\centering
\includegraphics{/Users/oda/Documents/iMyPRJ/CS132-Project-Banking-System/assets/image-20190513201400583.png}
\caption{}
\end{figure}

\begin{longtable}[]{@{}ll@{}}
\toprule
Seq No. in the use case & Counter part in implementation\tabularnewline
\midrule
\endhead
2 & Click button \texttt{Open\ Account}\tabularnewline
3 & Input at \texttt{Id.} and \texttt{Password} field\tabularnewline
3.1.1\textasciitilde{}3.1.3 & Creation of account is performed in
background\tabularnewline
3.1.4, 3.1.5 & The message at the center\tabularnewline
&\tabularnewline
\bottomrule
\end{longtable}

\textbf{Close account}. The following image is the use case of the
procedure close account, depicted in the requirement document.

\begin{figure}
\centering
\includegraphics{/Users/oda/Documents/iMyPRJ/CS132-Project-Banking-System/assets/image-20190513222359425.png}
\caption{}
\end{figure}

\begin{figure}
\centering
\includegraphics{/Users/oda/Documents/iMyPRJ/CS132-Project-Banking-System/assets/image-20190513201413946.png}
\caption{}
\end{figure}

\begin{longtable}[]{@{}ll@{}}
\toprule
Seq No. in the use case & Counter part in implementation\tabularnewline
\midrule
\endhead
2 & Click button \texttt{Close\ Account}\tabularnewline
2, 3 & The process is simplified to input all information at once. Input
person \texttt{identity\ number,\ password} and
\texttt{account\ number}.\tabularnewline
3.1.3 & The feedback is shown at the center.\tabularnewline
\bottomrule
\end{longtable}

\textbf{Change password}.

\begin{figure}
\centering
\includegraphics{/Users/oda/Documents/iMyPRJ/CS132-Project-Banking-System/assets/image-20190513222710709.png}
\caption{}
\end{figure}

\begin{figure}
\centering
\includegraphics{/Users/oda/Documents/iMyPRJ/CS132-Project-Banking-System/assets/image-20190513221903119.png}
\caption{}
\end{figure}

\begin{longtable}[]{@{}ll@{}}
\toprule
Seq No. in the use case & Counter part in implementation\tabularnewline
\midrule
\endhead
2 & Click button \texttt{Change\ Passsword}\tabularnewline
2, 3 & The process is simplified to input all information at once. Input
person \texttt{account\ number}, password\texttt{and}new
password`.\tabularnewline
3.1.3 & The feedback is shown at the center.\tabularnewline
\bottomrule
\end{longtable}

\textbf{Deposit}.

\begin{figure}
\centering
\includegraphics{/Users/oda/Documents/iMyPRJ/CS132-Project-Banking-System/assets/image-20190513223256136.png}
\caption{}
\end{figure}

\begin{figure}
\centering
\includegraphics{/Users/oda/Documents/iMyPRJ/CS132-Project-Banking-System/assets/image-20190513221455523.png}
\caption{}
\end{figure}

\begin{longtable}[]{@{}ll@{}}
\toprule
Seq No. in the use case & Counter part in implementation\tabularnewline
\midrule
\endhead
2 & Click button \texttt{Deposit}\tabularnewline
2, 3 & The process is simplified to input all information at once. Input
person \texttt{account\ number}, \texttt{Amount}.\tabularnewline
3.1.3 & The feedback is shown at the center.\tabularnewline
\bottomrule
\end{longtable}

\end{document}
